% Use extarticle for the smaller font size:
% https://tex.stackexchange.com/questions/5339/how-to-specify-font-size-less-than-10pt-or-more-than-12pt
\documentclass[a4paper, 8pt]{extarticle}

% Multi-column layout
\usepackage{multicol}

% Manually set page margins
\usepackage[margin=0.5cm]{geometry}

% Manually set margins on lists
\usepackage{enumitem}
% Change list margins - https://tex.stackexchange.com/questions/10684/vertical-space-in-lists
\setlist{leftmargin=3mm, nosep}
% or \setlist{noitemsep} to leave space around whole list
\usepackage{tikz}
\usetikzlibrary{shapes.geometric}
\usepackage{xcolor} % Paket für Farben
\usepackage{mathtools} % Für die 'dcases' Umgebung
\usepackage{rotating} % Paket für die Drehung
\usepackage{float}
\usepackage{amsmath}
\usepackage{mathtools}
\usepackage{amssymb}
\usepackage{graphicx}
\usepackage{pdflscape}
\usepackage{fancyhdr}
\usepackage{color}
\usepackage{amsmath}
\usepackage[T1]{fontenc}

\providecommand{\abs}[1]{\left\lvert#1\right\rvert}


\newcommand{\dx}{\hspace{2pt}dx} %% für saubere darstellung der tabelle
\newcommand{\dd}[1]{\hspace{2pt}d#1}

\begin{document}
\begin{landscape}
\section*{\hspace{10cm} Physik 1 Zusammenfassung MT/EU \hspace{8cm} \small Seite 1}
\section*{ \hspace{10cm} Autor: Michael Saxer DS22A}	% 4-column layout
	\begin{multicols*}{3}
\section{Statistische Masse und fehlerfortpflanzung}
n Messresultate \( x_1, \dots, x_n \), die sich aufgrund von statistischen Fehlern unterscheiden

\bigskip

\textbf{Mittelwert:}
\[
\bar{x} = \frac{1}{n} \sum_{i=1}^n x_i
\]

\bigskip

\textbf{Standardabweichung (Stichprobe):}
\[
\sigma = \sqrt{\frac{1}{n - 1} \sum_{i=1}^n (x_i - \bar{x})^2}
\]

\bigskip

\textbf{Absoluter Fehler des Mittelwertes:}
\[
\Delta = \frac{1}{\sqrt{n}} \cdot \sigma
\]
\subsection{REGELN:}

1.) Addition und Subtraktion: Die \textbf{absoluten} Messunsicherheiten addieren sich.
\[
\text{Bsp: } a = x + 3y - \frac{1}{6} z \rightarrow \Delta a = \Delta x + 3 \Delta y + \frac{1}{6} \Delta z
\]

2.) Multiplikation und Division: Die \textbf{relativen} Messunsicherheiten addieren sich.
\[
\text{Bsp: } a = \frac{3 \cdot x \cdot y}{z} \rightarrow \delta a = \delta x + \delta y + \delta z
\]

3.) Potenzierte Grössen: Die \textbf{relative} Messunsicherheit wird mit der Potenz multipliziert.
\[
\text{Bsp: } a = \frac{x \cdot y^5}{\sqrt{z}} \rightarrow \delta a = \delta x + 5 \cdot \delta y + \frac{1}{2} \delta z \quad \text{(Hinweis: } \sqrt{z} = z^{1/2}\text{)}
\]

\bigskip

\subsection{Beispielsaufgabe:} 
Die Dichte \( \rho = \frac{m}{V} \) eines Stoffes wird durch eine Messung der Masse \( m \) und eine Messung des Volumens \( V \) bestimmt, wobei \( m \) und \( V \) fehlerbehaftet sind. Wie gross ist der absolute Fehler der Dichte?

\[
m = (20.0 \pm 0.5) \, \text{kg} \quad \rightarrow \quad \delta m = 2.5\%
\]
\[
V = (5.0 \pm 0.2) \, \text{m}^3 \quad \rightarrow \quad \delta V = 4.0\%
\]

\[
\rho = \frac{m}{V} = \frac{20 \, \text{kg}}{5 \, \text{m}^3} = 4 \, \text{kg/m}^3
\]

\[
\delta \rho = \delta m + \delta V = 6.5\% \quad \rightarrow \quad \Delta \rho = \delta \rho \cdot \rho = 0.065 \cdot 4 \, \text{kg/m}^3 = 0.26 \, \text{kg/m}^3
\]

\[
\rho = (4.0 \pm 0.26) \, \text{kg/m}^3
\]

\section{Erhaltungsgrösse Energie}
\begin{itemize}
    \item Energie
    \item Impuls
    \item Drehimpuls
\end{itemize}

\bigskip

\textit{Bemerkung:} \\
Es gibt auch noch drei weitere Erhaltungsgrössen in der Physik: \\
Elektrische Ladung, Baryonenzahl, Leptonenzahl

\bigskip

\textbf{\underline{Extensive}} \textbf{Zustandsgrössen} sind mengenartig, d.h. sie ändern sich mit der Grösse des betrachteten Systems (z.B: Masse, Volumen, Energie, \dots)

\bigskip

\textbf{\underline{Intensive}} \textbf{Zustandsgrössen} ändern sich nicht mit der Grösse des betrachteten Systems (z.B.: Temperatur, Druck, Geschwindigkeit, \dots)

\bigskip

\fbox{\underline{\textbf{Erhaltungsgrössen sind extensive Zustandsgrössen}}}
\section*{Energieformen}

\subsection{ Kinetische Energie}

\begin{itemize}
    \item Man braucht Kraft, um ein "Objekt schnell zu machen".
    \item Die Energie, die man durch Ziehen in das Objekt hineinbringt, ist dann im Objekt gespeichert.
    \item Ein Objekt mit Masse \( m \) und Geschwindigkeit \( v \) beinhaltet die kinetische Energie.
\end{itemize}

\[
E_{\text{kin}} = \frac{mv^2}{2}
\]

\subsection{ Potenzielle Energie im Gravitationsfeld}

Ein Objekt mit Masse \( m \) im Gravitationsfeld der Erde mit der Erdbeschleunigung \( g \) hat in der Höhe \( h \) die potenzielle Energie:

\[
E_{\text{pot}} = mgh
\]

\textit{Achtung:} Die potenzielle Energie ist eigentlich eine Energiedifferenz. Es gibt eine willkürliche Festlegung eines Nullpunktes mit der Höhe \( h = 0 \). Um eine Masse \( m \) von einer Höhe \( h_1 \) auf eine Höhe \( h_2 \) zu bringen, braucht man die Energie:

\[
E_{\text{pot}} = mg(h_2 - h_1)
\]

\subsection{ Federenergie}

Die Energieberechnung für eine Feder geht wie folgt:

\[
E_{\text{Feder}} = \frac{k (x - L)^2}{2}
\]

\textbf{Legende:}
\begin{itemize}
    \item \( L \): Ruhelänge der Feder
    \item \( x \): Länge der Feder im ausgelenkten Zustand
    \item \( k \): Federkonstante [N/m]
\end{itemize}

\subsection{ Weitere Energieformen}

\begin{itemize}
    \item Rotationsenergie
    \item Verschiebungsenergie
    \item Wärmeenergie
    \item Elektrische Energie
    \item Magnetische Energie
    \item Chemische Energie
    \item \dots
\end{itemize}

\section{Erhaltungsgrösse Impuls}
\subsection{ Definition Impuls}

\[
\vec{p} = m \cdot \vec{v}_s
\]

\begin{itemize}
    \item \( \vec{p} \): Impuls des Körpers; Einheit: \( \text{kg} \cdot \frac{\text{m}}{\text{s}} = \text{N} \cdot \text{s} \)
    \item \( m \): Masse des Körpers; Einheit: \( \text{kg} \)
    \item \( \vec{v}_s \): Schwerpunktgeschwindigkeit des Körpers; Einheit: \( \frac{\text{m}}{\text{s}} \)
\end{itemize}

\bigskip

\[
\vec{p} = \begin{pmatrix} p_x \\ p_y \\ p_z \end{pmatrix} = m \begin{pmatrix} v_{sx} \\ v_{sy} \\ v_{sz} \end{pmatrix}
\]

\textit{Bemerkung:} Die Vektorkomponenten werden immer bezüglich eines orthogonalen Koordinatensystems \( (x, y, z) \) angegeben, das frei definiert werden kann.

\subsubsection{ Gesamterhaltungsimpuls des Systems}

\[
\vec{p}_{\text{tot}} = \sum_i \vec{p}_i = \text{const.}
\]

\begin{itemize}
    \item \( \vec{p}_{\text{tot}} \): Gesamtimpuls des Systems
    \item \( \vec{p}_i \): Impuls eines einzelnen Körpers des Systems
\end{itemize}


\subsection{ Schwerpunktgeschwindigkeit}

Ein System besteht aus mehreren Massen \( m_i \) mit den Impulsen \( \vec{p}_i \). Auf das System wirken keine äußeren Kräfte, oder die Summe der äußeren Kräfte ist gleich null. Es gilt:

\[
\vec{p}_{\text{tot}} = \sum_i \vec{p}_i = \sum_i m_i \cdot \vec{v}_i = \sum_i m_i \cdot \vec{v}_i = M \cdot \vec{v}_s = \text{const.}
\]

\begin{itemize}
    \item \( \vec{p}_{\text{tot}} \): Gesamtimpuls des Systems
    \item \( m_i \): Masse des \( i \)-ten Körpers im System
    \item \( \vec{v}_i \): Geschwindigkeit des \( i \)-ten Körpers im System
    \item \( M = \sum_i m_i \): Gesamtmasse des Systems
    \item \( \vec{v}_s = \frac{\sum_i m_i \cdot \vec{v}_i}{\sum_i m_i} \): Schwerpunktgeschwindigkeit des Systems
\end{itemize}

\subsection{ Bestimmung des Schwerpunkts von mehreren Massen}

Der Ortsvektor \( \vec{r}_s \) des Schwerpunkts eines Systems aus mehreren Massen \( m_i \) mit den Ortsvektoren \( \vec{r}_i \) ist gegeben durch:

\[
\vec{r}_s = \frac{\sum_i m_i \cdot \vec{r}_i}{\sum_i m_i}
\]

\begin{itemize}
    \item \( \vec{r}_s \): Ortsvektor des Schwerpunkts
    \item \( m_i \): Masse des \( i \)-ten Körpers im System
    \item \( \vec{r}_i \): Ortsvektor des \( i \)-ten Körpers im System
\end{itemize}

\subsubsection{Beispiel:} Drei Punktmassen haben die Koordinaten:
\begin{itemize}
    \item \( m_1 = 1 \, \text{kg} \), \( \vec{r}_1 = (0.0, 0.0, 0.0) \, \text{m} \)
    \item \( m_2 = 2 \, \text{kg} \), \( \vec{r}_2 = (1.0, 0.0, 0.0) \, \text{m} \)
    \item \( m_3 = 4 \, \text{kg} \), \( \vec{r}_3 = (0.2, 0.0, 0.0) \, \text{m} \)
\end{itemize}

Der Ortsvektor des Schwerpunkts der drei Punktmassen \( \vec{r}_s \) ist:

\[
\vec{r}_s = \frac{1}{(1 \, \text{kg} + 2 \, \text{kg} + 4 \, \text{kg})} \begin{pmatrix} 1 \cdot 0 + 2 \cdot 1 + 4 \cdot 0.2 \\ 1 \cdot 0 + 2 \cdot 0 + 4 \cdot 0 \\ 1 \cdot 0 + 2 \cdot 0 + 4 \cdot 0 \end{pmatrix} = \begin{pmatrix} \frac{2}{7} \\ \frac{8}{7} \\ 0 \end{pmatrix} \, \text{m}
\]
\subsection{Stöße}

\subsubsection{Vollkommen elastischer Stoß}

\textit{Die innere Energie der Stoßpartner vor und nach dem Stoß bleibt unverändert}. Bei einem solchen Stoß in einer horizontalen Ebene bleibt somit die kinetische Energie „vor“ und „nach“ dem Stoß erhalten (Beispiel: Billardkugeln).


\subsection{Vollkommen elastischer zentraler Stoß}

Beim vollkommen elastischen zentralen Stoß bleibt sowohl der Gesamtimpuls als auch die kinetische Energie erhalten.

\subsubsection{Vor dem Stoß}
\[
p_{\text{tot},x}^I = m_1 \cdot v_1 + m_2 \cdot v_2
\]
{\footnotesize
\begin{itemize}
    \item \( p_{\text{tot},x}^I \): Gesamtimpuls des Systems in x-Richtung vor dem Stoß
    \item \( m_1 \): Masse des ersten Stoßpartners
    \item \( v_1 \): Geschwindigkeit des ersten Stoßpartners vor dem Stoß
    \item \( m_2 \): Masse des zweiten Stoßpartners
    \item \( v_2 \): Geschwindigkeit des zweiten Stoßpartners vor dem Stoß
\end{itemize}
}

\[
E_{\text{kin, tot}}^I = \frac{m_1 v_1^2}{2} + \frac{m_2 v_2^2}{2}
\]
{\footnotesize
\begin{itemize}
    \item \( E_{\text{kin, tot}}^I \): Gesamte kinetische Energie des Systems vor dem Stoß
    \item \( m_1 \): Masse des ersten Stoßpartners
    \item \( v_1 \): Geschwindigkeit des ersten Stoßpartners vor dem Stoß
    \item \( m_2 \): Masse des zweiten Stoßpartners
    \item \( v_2 \): Geschwindigkeit des zweiten Stoßpartners vor dem Stoß
\end{itemize}
}

\subsubsection{Nach dem Stoß}
\[
p_{\text{tot},x}^{II} = m_1 \cdot \hat{v}_1 + m_2 \cdot \hat{v}_2
\]
{\footnotesize
\begin{itemize}
    \item \( p_{\text{tot},x}^{II} \): Gesamtimpuls des Systems in x-Richtung nach dem Stoß
    \item \( m_1 \): Masse des ersten Stoßpartners
    \item \( \hat{v}_1 \): Geschwindigkeit des ersten Stoßpartners nach dem Stoß
    \item \( m_2 \): Masse des zweiten Stoßpartners
    \item \( \hat{v}_2 \): Geschwindigkeit des zweiten Stoßpartners nach dem Stoß
\end{itemize}
}

\[
E_{\text{kin, tot}}^{II} = \frac{m_1 \hat{v}_1^2}{2} + \frac{m_2 \hat{v}_2^2}{2}
\]
{\footnotesize
\begin{itemize}
    \item \( E_{\text{kin, tot}}^{II} \): Gesamte kinetische Energie des Systems nach dem Stoß
    \item \( m_1 \): Masse des ersten Stoßpartners
    \item \( \hat{v}_1 \): Geschwindigkeit des ersten Stoßpartners nach dem Stoß
    \item \( m_2 \): Masse des zweiten Stoßpartners
    \item \( \hat{v}_2 \): Geschwindigkeit des zweiten Stoßpartners nach dem Stoß
\end{itemize}
}

\textbf{Erhaltungsgleichungen}:
\[
\hat{v}_1 = \frac{(m_1 - m_2) \cdot v_1 + 2 m_2 \cdot v_2}{m_1 + m_2}
\]
{\footnotesize
\begin{itemize}
    \item \( \hat{v}_1 \): Geschwindigkeit des ersten Stoßpartners nach dem Stoß
    \item \( m_1 \): Masse des ersten Stoßpartners
    \item \( m_2 \): Masse des zweiten Stoßpartners
    \item \( v_1 \): Geschwindigkeit des ersten Stoßpartners vor dem Stoß
    \item \( v_2 \): Geschwindigkeit des zweiten Stoßpartners vor dem Stoß
\end{itemize}
}

\[
\hat{v}_2 = \frac{(m_2 - m_1) \cdot v_2 + 2 m_1 \cdot v_1}{m_1 + m_2}
\]
{\footnotesize
\begin{itemize}
    \item \( \hat{v}_2 \): Geschwindigkeit des zweiten Stoßpartners nach dem Stoß
    \item \( m_1 \): Masse des ersten Stoßpartners
    \item \( m_2 \): Masse des zweiten Stoßpartners
    \item \( v_1 \): Geschwindigkeit des ersten Stoßpartners vor dem Stoß
    \item \( v_2 \): Geschwindigkeit des zweiten Stoßpartners vor dem Stoß
\end{itemize}
}

\subsubsection{Geschwindigkeitsdiagramm}
Im \( v(t) \)-Diagramm für einen vollkommen elastischen Stoß bleibt die Schwerpunktgeschwindigkeit \( v_s \) konstant. Die Relation lautet:
\[
v_s = \frac{v_1 + \hat{v}_1}{2} = \frac{v_2 + \hat{v}_2}{2}
\]
{\footnotesize
\begin{itemize}
    \item \( v_s \): Schwerpunktgeschwindigkeit des Systems
    \item \( v_1 \): Geschwindigkeit des ersten Stoßpartners vor dem Stoß
    \item \( \hat{v}_1 \): Geschwindigkeit des ersten Stoßpartners nach dem Stoß
    \item \( v_2 \): Geschwindigkeit des zweiten Stoßpartners vor dem Stoß
    \item \( \hat{v}_2 \): Geschwindigkeit des zweiten Stoßpartners nach dem Stoß
\end{itemize}
}

\textbf{Hinweis:} Kennt man die Geschwindigkeit eines Körpers vor und nach dem Stoß, lässt sich die Schwerpunktgeschwindigkeit berechnen.











\subsubsection{Vollkommen inelastischer Stoß}

Beide stoßenden Körper bewegen sich nach dem Stoß gemeinsam weiter und haben somit dieselbe Geschwindigkeit. Die innere Energie der Stoßpartner ändert sich \textit{maximal}, d.h. soweit es der Impulserhaltungssatz zulässt. Es wird Energie dissipiert oder in der Verformung „gespeichert“. In der Praxis wird hauptsächlich Wärme freigesetzt (Beispiel: Schneeball auf Auto werfen).
\[
\hat{v} = \frac{m_1 \cdot v_1 + m_2 \cdot v_2}{m_1 + m_2}
\]

\begin{itemize}
    \item \( \hat{v} \): Geschwindigkeit nach dem Stoß, die der Schwerpunktgeschwindigkeit entspricht
    \item \( m_1 \): Masse des ersten Stoßpartners
    \item \( v_1 \): Geschwindigkeit des ersten Stoßpartners vor dem Stoß
    \item \( m_2 \): Masse des zweiten Stoßpartners
    \item \( v_2 \): Geschwindigkeit des zweiten Stoßpartners vor dem Stoß
\end{itemize}

Die Geschwindigkeit nach dem Stoß entspricht der Schwerpunktgeschwindigkeit. Diese bleibt bei dem Stoß erhalten, weil es keine äußeren Kräfte gibt.
\subsection{Energie beim vollkommen inelastischen zentralen Stoß}

Zwei Körper mit gleichen Massen \( m_1 = m_2 = m \) und entgegengesetzten Geschwindigkeiten \( v_2 = -v_1 = -v \) stoßen vollkommen inelastisch zusammen. Nach dem Stoß bewegen sich die Körper gemeinsam weiter oder kommen zum Stillstand.

\subsubsection{Vor dem Stoß}
\begin{itemize}
    \item Gesamtimpuls in x-Richtung: 
    \[
    p_{\text{tot}, x} = m \cdot v + m \cdot (-v) = 0
    \]
    \item Gesamte kinetische Energie: 
    \[
    E_{\text{kin, tot}}^I = \frac{m v^2}{2} + \frac{m v^2}{2} = m v^2 > 0
    \]
\end{itemize}

\subsubsection{Nach dem Stoß}
\begin{itemize}
    \item Gesamtimpuls in x-Richtung bleibt \( 0 \), und die gemeinsame Geschwindigkeit \( \hat{v} = 0 \).
    \item Gesamte kinetische Energie: 
    \[
    E_{\text{kin, tot}}^{II} = \frac{(m + m) \cdot \hat{v}^2}{2} = 0
    \]
\end{itemize}

\textit{Fazit:} Die kinetische Energie geht bei einem vollkommen inelastischen Stoß teilweise oder vollständig in innere Energie über (z.B. Wärme).

\section{Kinematik}

\subsection{Ortsvektor und Geschwindigkeit}

\subsubsection{Ortsvektor}
\[
\vec{r}(t) = \begin{pmatrix} x(t) \\ y(t) \\ z(t) \end{pmatrix}
\]
{\footnotesize
\begin{itemize}
    \item \( \vec{r}(t) \): Ortsvektor in drei Dimensionen als Funktion der Zeit \( t \)
    \item \( x(t), y(t), z(t) \): Komponenten des Ortsvektors in x-, y- und z-Richtung
\end{itemize}
}

\subsubsection{Betrag des Ortsvektors}
\[
|\vec{r}(t)| = \sqrt{x^2(t) + y^2(t) + z^2(t)}
\]
{\footnotesize
\begin{itemize}
    \item \( |\vec{r}(t)| \): Betrag (Länge) des Ortsvektors
    \item \( x(t), y(t), z(t) \): Komponenten des Ortsvektors in x-, y- und z-Richtung
\end{itemize}
}

\subsubsection{Mittlere Geschwindigkeit (1-dimensional)}
\[
\bar{v}_x = \frac{\Delta x}{\Delta t} = \frac{x(t_2) - x(t_1)}{t_2 - t_1} = \frac{x(t_1 + \Delta t) - x(t_1)}{\Delta t}
\]
{\footnotesize
\begin{itemize}
    \item \( \bar{v}_x \): Mittlere Geschwindigkeit in x-Richtung
    \item \( \Delta x \): Differenz des Ortes \( x \) zwischen zwei Zeitpunkten
    \item \( \Delta t = t_2 - t_1 \): Zeitintervall
    \item \( x(t_1), x(t_2) \): Ort zur Zeit \( t_1 \) und \( t_2 \)
\end{itemize}
}

\subsubsection{Momentane Geschwindigkeit (1-dimensional)}
\[
v_x(t) = \lim_{\Delta t \to 0} \frac{\Delta x}{\Delta t} = \frac{dx}{dt} = \dot{x}(t)
\]
{\footnotesize
\begin{itemize}
    \item \( v_x(t) \): Momentane Geschwindigkeit in x-Richtung zur Zeit \( t \)
    \item \( \Delta x \): Änderung des Ortes \( x \) in einem kleinen Zeitintervall \( \Delta t \)
    \item \( \frac{dx}{dt} \): Ableitung des Ortes nach der Zeit (Geschwindigkeit)
\end{itemize}
}

\subsubsection{Momentane Geschwindigkeit (3-dimensional)}
\[
\vec{v}(t) = \lim_{\Delta t \to 0} \frac{\Delta \vec{r}}{\Delta t} = \frac{d\vec{r}}{dt} = \dot{\vec{r}} = \begin{pmatrix} v_x(t) \\ v_y(t) \\ v_z(t) \end{pmatrix} = \begin{pmatrix} \frac{dx}{dt} \\ \frac{dy}{dt} \\ \frac{dz}{dt} \end{pmatrix}
\]
{\footnotesize
\begin{itemize}
    \item \( \vec{v}(t) \): Momentane Geschwindigkeit als Vektor zur Zeit \( t \)
    \item \( \Delta \vec{r} \): Änderung des Ortsvektors \( \vec{r} \) in einem kleinen Zeitintervall \( \Delta t \)
    \item \( \frac{d\vec{r}}{dt} \): Ableitung des Ortsvektors nach der Zeit (Geschwindigkeitsvektor)
    \item \( v_x(t), v_y(t), v_z(t) \): Komponenten der Geschwindigkeit in x-, y- und z-Richtung
\end{itemize}
}

\subsubsection{Betrag der Geschwindigkeit}
\[
|\vec{v}(t)| = \sqrt{v_x^2(t) + v_y^2(t) + v_z^2(t)}
\]
{\footnotesize
\begin{itemize}
    \item \( |\vec{v}(t)| \): Betrag der Geschwindigkeit (auch Schnelligkeit genannt)
    \item \( v_x(t), v_y(t), v_z(t) \): Komponenten der Geschwindigkeit in x-, y- und z-Richtung
\end{itemize}
}

\subsection{Integration der Geschwindigkeit und Beschleunigung (3-dimensional)}

\subsubsection{Ortsvektor durch Integration der Geschwindigkeit}
\[
\vec{r}(t) = \vec{r}(t_0) + \int_{t_0}^{t} \vec{v}(\tilde{t}) \, d\tilde{t} = \begin{pmatrix} x(t_0) + \int_{t_0}^{t} v_x(\tilde{t}) \, d\tilde{t} \\ y(t_0) + \int_{t_0}^{t} v_y(\tilde{t}) \, d\tilde{t} \\ z(t_0) + \int_{t_0}^{t} v_z(\tilde{t}) \, d\tilde{t} \end{pmatrix}
\]
{\footnotesize
\begin{itemize}
    \item \( \vec{r}(t) \): Ortsvektor zur Zeit \( t \)
    \item \( \vec{r}(t_0) \): Anfangsortsvektor zur Startzeit \( t_0 \)
    \item \( \vec{v}(\tilde{t}) \): Geschwindigkeitsvektor als Funktion der Integrationsvariablen \( \tilde{t} \)
    \item \( x(t_0), y(t_0), z(t_0) \): Anfangspositionen in x-, y- und z-Richtung
    \item \( v_x(\tilde{t}), v_y(\tilde{t}), v_z(\tilde{t}) \): Komponenten der Geschwindigkeit in x-, y- und z-Richtung als Funktionen von \( \tilde{t} \)
\end{itemize}
}

\subsubsection{Geschwindigkeitsvektor durch Integration der Beschleunigung}
\[
\vec{v}(t) = \vec{v}(t_0) + \int_{t_0}^{t} \vec{a}(\tilde{t}) \, d\tilde{t} = \begin{pmatrix} v_x(t_0) + \int_{t_0}^{t} a_x(\tilde{t}) \, d\tilde{t} \\ v_y(t_0) + \int_{t_0}^{t} a_y(\tilde{t}) \, d\tilde{t} \\ v_z(t_0) + \int_{t_0}^{t} a_z(\tilde{t}) \, d\tilde{t} \end{pmatrix}
\]
{\footnotesize
\begin{itemize}
    \item \( \vec{v}(t) \): Geschwindigkeitsvektor zur Zeit \( t \)
    \item \( \vec{v}(t_0) \): Anfangsgeschwindigkeitsvektor zur Startzeit \( t_0 \)
    \item \( \vec{a}(\tilde{t}) \): Beschleunigungsvektor als Funktion der Integrationsvariablen \( \tilde{t} \)
    \item \( v_x(t_0), v_y(t_0), v_z(t_0) \): Anfangsgeschwindigkeiten in x-, y- und z-Richtung
    \item \( a_x(\tilde{t}), a_y(\tilde{t}), a_z(\tilde{t}) \): Komponenten der Beschleunigung in x-, y- und z-Richtung als Funktionen von \( \tilde{t} \)
\end{itemize}
}



\section{Bilanzen}

\subsection{Volumenbilanz in der Hydraulik}
\[
\frac{dV(t)}{dt} = I_v(t)
\]
\begin{itemize}
    \item \( V(t) \): Volumen des Systems zur Zeit \( t \)
    \item \( I_v(t) \): Volumenstrom (Flussrate) in \( m^3/s \)
\end{itemize}

\subsection{Momentanbilanz bei mehreren Zu- und Abflüssen}
\[
\frac{dV(t)}{dt} = \sum_{i=1}^n I_{v,i}(t)
\]
\begin{itemize}
    \item \( I_{v,i}(t) \): Volumenstrom des \( i \)-ten Flusses
\end{itemize}

\subsection{Energiebilanz}
\[
\frac{dU(t)}{dt} = \sum_{i=1}^n I_{W,i}(t)
\]
\begin{itemize}
    \item \( U(t) \): Innere Energie des Systems
    \item \( I_{W,i}(t) \): Energiestrom \( i \) (Zufluss oder Abfluss von Energie)
\end{itemize}

\subsection{Leistung und Energiestrom}
\[
P = \frac{dE}{dt} = I_W
\]
\begin{itemize}
    \item \( P \): Leistung, auch als Energiestrom bezeichnet
    \item \( E \): Energie des Systems
\end{itemize}

\subsection{Wirkungsgrad}
\[
\eta = \frac{\text{gewünschte Leistung}}{\text{eingebrachte Leistung}}
\]
\begin{itemize}
    \item \( \eta \): Wirkungsgrad (zwischen 0 und 1)
\end{itemize}

\subsection{Impulsbilanz}
\[
\frac{d\vec{p}_{\text{tot}}(t)}{dt} = \sum_{i=1}^n I_{p,i}(t)
\]
\begin{itemize}
    \item \( \vec{p}_{\text{tot}}(t) \): Gesamtimpuls des Systems
    \item \( I_{p,i}(t) \): Impulsstrom \( i \)
\end{itemize}
\section{Energie, Impulsströme & Kraft}

\subsection{Kinetische Energie (1-d)}
\[
E_{\text{kin}} = \frac{p_x^2}{2m} = \frac{m \cdot v_x^2}{2}
\]
{\footnotesize
\begin{itemize}
    \item \( E_{\text{kin}} \): Kinetische Energie des Körpers
    \item \( p_x \): Impuls des Körpers in x-Richtung
    \item \( m \): Masse des Körpers
    \item \( v_x \): Geschwindigkeit in x-Richtung
\end{itemize}
}

\subsection{Änderung der kinetischen Energie und Impulsstrom}
\[
\frac{dE_{\text{kin}}}{dt} = v_x \cdot I_{p_x}
\]
{\footnotesize
\begin{itemize}
    \item \( \frac{dE_{\text{kin}}}{dt} \): Zeitliche Änderung der kinetischen Energie
    \item \( v_x \): Geschwindigkeit in x-Richtung
    \item \( I_{p_x} \): Impulsstrom in x-Richtung
\end{itemize}
}

\subsection{Potentielle Energie im Gravitationsfeld (1-d)}
\[
E_{\text{pot}} = m \cdot g \cdot z
\]
{\footnotesize
\begin{itemize}
    \item \( E_{\text{pot}} \): Potentielle Energie des Körpers
    \item \( m \): Masse des Körpers
    \item \( g \): Erdbeschleunigung
    \item \( z \): Höhe des Körpers im Bezugssystem
\end{itemize}
}

\subsection{Änderung der potentiellen Energie}
\[
\frac{dE_{\text{pot}}}{dt} = -F_z \cdot v_z
\]
{\footnotesize
\begin{itemize}
    \item \( \frac{dE_{\text{pot}}}{dt} \): Zeitliche Änderung der potentiellen Energie
    \item \( F_z \): Kraft in z-Richtung
    \item \( v_z \): Geschwindigkeit in z-Richtung
\end{itemize}
}

\subsection{Energieerhaltung im freien Fall}
\[
\frac{p_1^2}{2m} + m \cdot g \cdot z_1 = \frac{p_2^2}{2m} + m \cdot g \cdot z_2
\]
{\footnotesize
\begin{itemize}
    \item \( p_1, p_2 \): Impulse des Körpers an den Positionen \( z_1 \) und \( z_2 \)
    \item \( z_1, z_2 \): Höhen des Körpers an zwei verschiedenen Zeitpunkten
\end{itemize}
}

\subsection{Newtonsche Kraft und Impulsstrom}
\[
\frac{d\vec{p}}{dt} = \vec{F}
\]
{\footnotesize
\begin{itemize}
    \item \( \frac{d\vec{p}}{dt} \): Änderungsrate des Impulses (Impulsstrom)
    \item \( \vec{F} \): Kraft
\end{itemize}
}

\subsection{Bewegungsgleichung eines Körpers im Gravitationsfeld (3-d)}
\[
\vec{F} = m \cdot \vec{a}
\]
{\footnotesize
\begin{itemize}
    \item \( \vec{F} \): Gesamtkraft auf den Körper
    \item \( m \): Masse des Körpers
    \item \( \vec{a} \): Beschleunigung des Körpers
\end{itemize}
}
\section{Reibung
\& offene Systeme}

\subsection{Arbeit entlang eines Weges}
\[
W_{AB} = \int_A^B \vec{F} \cdot d\vec{r} = \int_{t_A}^{t_B} \vec{F} \cdot \vec{v} \, dt
\]
{\footnotesize
\begin{itemize}
    \item \( W_{AB} \): Arbeit, die durch die Kraft \( \vec{F} \) entlang des Weges von \( A \) nach \( B \) verrichtet wird
    \item \( \vec{F} \): Wirksame Kraft entlang des Weges
    \item \( \vec{r} \): Ortsvektor
    \item \( \vec{v} \): Geschwindigkeit
\end{itemize}
}

\subsection{Arbeit einer konstanten Kraft entlang einer geraden Bahn}
\[
W_{AB} = F \cdot \cos(\alpha) \cdot s_{AB}
\]
{\footnotesize
\begin{itemize}
    \item \( F \): Betrag der konstanten Kraft
    \item \( \alpha \): Winkel zwischen der Kraft und der geradlinigen Bahn
    \item \( s_{AB} \): Distanz zwischen den Punkten \( A \) und \( B \)
\end{itemize}
}

\subsection{Energiestrom im offenen System}
\[
\frac{dE}{dt} = \vec{F}_{\text{ext}} \cdot \vec{v} = I_{\text{W ext}}
\]
{\footnotesize
\begin{itemize}
    \item \( \frac{dE}{dt} \): Energiestrom (zeitliche Änderung der Energie)
    \item \( \vec{F}_{\text{ext}} \): Externe Kraft, die auf das System wirkt
    \item \( \vec{v} \): Geschwindigkeit des Körpers
    \item \( I_{\text{W ext}} \): Energiestrom durch die Systemgrenze
\end{itemize}
}

\subsection{Impulsstrom im offenen System}
\[
\frac{d\vec{p}}{dt} = \vec{F}_{\text{pot}} + \vec{F}_{\text{ext}}
\]
{\footnotesize
\begin{itemize}
    \item \( \frac{d\vec{p}}{dt} \): Zeitliche Änderung des Impulses (Impulsstrom)
    \item \( \vec{F}_{\text{pot}} \): Potentialkraft
    \item \( \vec{F}_{\text{ext}} \): Externe Kraft
\end{itemize}
}

\subsection{Luftwiderstandskraft}
\[
\vec{F}_R = -\frac{1}{2} \rho \cdot v^2 \cdot A \cdot c_w \cdot \frac{\vec{v}}{|\vec{v}|}
\]
{\footnotesize
\begin{itemize}
    \item \( \vec{F}_R \): Luftwiderstandskraft
    \item \( \rho \): Dichte des Mediums (z.B. Luft)
    \item \( v \): Betrag der Geschwindigkeit
    \item \( A \): Stirnfläche des Körpers senkrecht zur Geschwindigkeit
    \item \( c_w \): Widerstandsbeiwert
\end{itemize}
}

\subsection{Maximale Geschwindigkeit beim vertikalen Wurf mit Luftwiderstand}
\[
v_{\infty} = \sqrt{\frac{2 \cdot m \cdot g}{\rho \cdot A \cdot c_w}}
\]
{\footnotesize
\begin{itemize}
    \item \( v_{\infty} \): Maximale Geschwindigkeit des Körpers
    \item \( m \): Masse des Körpers
    \item \( g \): Erdbeschleunigung
    \item \( \rho \): Dichte des Mediums (z.B. Luft)
    \item \( A \): Stirnfläche des Körpers senkrecht zur Geschwindigkeit
    \item \( c_w \): Widerstandsbeiwert
\end{itemize}
}
\section{Kraftstoss & Impuls}

\subsection{Impulsänderung durch Kraftstoß}
\[
\Delta \vec{p} = \int_{t_A}^{t_B} \vec{F} \, dt
\]
{\footnotesize
\begin{itemize}
    \item \( \Delta \vec{p} \): Änderung des Impulses zwischen den Zeitpunkten \( t_A \) und \( t_B \)
    \item \( \vec{F} \): Kraft, die auf den Körper wirkt
\end{itemize}
}

\subsection{Impulsänderung bei konstanter Kraft (1-d)}
\[
\Delta p_{AB} = F_{\text{const}} \cdot \Delta t_{AB}
\]
{\footnotesize
\begin{itemize}
    \item \( \Delta p_{AB} \): Impulsänderung zwischen den Punkten \( A \) und \( B \)
    \item \( F_{\text{const}} \): Konstante Kraft
    \item \( \Delta t_{AB} \): Zeitdauer zwischen \( t_A \) und \( t_B \)
\end{itemize}
}

\subsection{Federkraft und Federenergie}
\[
F = -k \cdot (L - x)
\]
\[
E_{\text{Feder}} = \frac{1}{2} k \cdot (L - x)^2
\]
{\footnotesize
\begin{itemize}
    \item \( F \): Federkraft
    \item \( k \): Federkonstante
    \item \( L \): Ruhelänge der Feder
    \item \( x \): Aktuelle Länge der Feder
    \item \( E_{\text{Feder}} \): In der Feder gespeicherte Energie
\end{itemize}
}

\subsection{Elastischer Stoß mit maximaler Verformung}
\[
\Delta x_{\text{max}} = \frac{p_0}{m \cdot \omega}
\]
{\footnotesize
\begin{itemize}
    \item \( \Delta x_{\text{max}} \): Maximale Verformung der Feder beim Stoß
    \item \( p_0 \): Anfangsimpuls
    \item \( m \): Masse des Körpers
    \item \( \omega = \sqrt{\frac{k}{m}} \): Kreisfrequenz der Schwingung
\end{itemize}
}

\subsection{Stoßdauer und maximale Stoßkraft}
\[
\Delta t_{\text{Stoß}} = \frac{\pi}{\omega}
\]
\[
F_{\text{max}} = k \cdot \Delta x_{\text{max}}
\]
{\footnotesize
\begin{itemize}
    \item \( \Delta t_{\text{Stoß}} \): Dauer des Stoßes
    \item \( F_{\text{max}} \): Maximale Stoßkraft
\end{itemize}
}


		\end{multicols*}{3}

\end{landscape}
\end{document}