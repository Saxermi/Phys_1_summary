% Use extarticle for the smaller font size:
% https://tex.stackexchange.com/questions/5339/how-to-specify-font-size-less-than-10pt-or-more-than-12pt
\documentclass[a4paper, 8pt]{extarticle}

% Multi-column layout
\usepackage{multicol}

% Manually set page margins
\usepackage[margin=0.5cm]{geometry}

% Manually set margins on lists
\usepackage{enumitem}
% Change list margins - https://tex.stackexchange.com/questions/10684/vertical-space-in-lists
\setlist{leftmargin=3mm, nosep}
% or \setlist{noitemsep} to leave space around whole list
\usepackage{tikz}
\usetikzlibrary{shapes.geometric}
\usepackage{xcolor} % Paket für Farben
\usepackage{mathtools} % Für die 'dcases' Umgebung
\usepackage{rotating} % Paket für die Drehung
\usepackage{float}
\usepackage{amsmath}
\usepackage{mathtools}
\usepackage{amssymb}
\usepackage{graphicx}
\usepackage{pdflscape}
\usepackage{fancyhdr}
\usepackage{color}
\usepackage{amsmath}
\providecommand{\abs}[1]{\left\lvert#1\right\rvert}


\newcommand{\dx}{\hspace{2pt}dx} %% für saubere darstellung der tabelle
\newcommand{\dd}[1]{\hspace{2pt}d#1}

\begin{document}
\begin{landscape}
\section*{\hspace{10cm} Physik 1 Zusammenfassung MT/EU \hspace{8cm} \small Seite 1}
\section*{ \hspace{10cm} Autor: Michael Saxer DS22A}	% 4-column layout
	\begin{multicols*}{3}
\section{Statistische Masse und fehlerfortpflanzung}
n Messresultate \( x_1, \dots, x_n \), die sich aufgrund von statistischen Fehlern unterscheiden

\bigskip

\textbf{Mittelwert:}
\[
\bar{x} = \frac{1}{n} \sum_{i=1}^n x_i
\]

\bigskip

\textbf{Standardabweichung (Stichprobe):}
\[
\sigma = \sqrt{\frac{1}{n - 1} \sum_{i=1}^n (x_i - \bar{x})^2}
\]

\bigskip

\textbf{Absoluter Fehler des Mittelwertes:}
\[
\Delta = \frac{1}{\sqrt{n}} \cdot \sigma
\]
\subsection{REGELN:}

1.) Addition und Subtraktion: Die \textbf{absoluten} Messunsicherheiten addieren sich.
\[
\text{Bsp: } a = x + 3y - \frac{1}{6} z \rightarrow \Delta a = \Delta x + 3 \Delta y + \frac{1}{6} \Delta z
\]

2.) Multiplikation und Division: Die \textbf{relativen} Messunsicherheiten addieren sich.
\[
\text{Bsp: } a = \frac{3 \cdot x \cdot y}{z} \rightarrow \delta a = \delta x + \delta y + \delta z
\]

3.) Potenzierte Grössen: Die \textbf{relative} Messunsicherheit wird mit der Potenz multipliziert.
\[
\text{Bsp: } a = \frac{x \cdot y^5}{\sqrt{z}} \rightarrow \delta a = \delta x + 5 \cdot \delta y + \frac{1}{2} \delta z \quad \text{(Hinweis: } \sqrt{z} = z^{1/2}\text{)}
\]

\bigskip

\subsection{Beispielsaufgabe:} 
Die Dichte \( \rho = \frac{m}{V} \) eines Stoffes wird durch eine Messung der Masse \( m \) und eine Messung des Volumens \( V \) bestimmt, wobei \( m \) und \( V \) fehlerbehaftet sind. Wie gross ist der absolute Fehler der Dichte?

\[
m = (20.0 \pm 0.5) \, \text{kg} \quad \rightarrow \quad \delta m = 2.5\%
\]
\[
V = (5.0 \pm 0.2) \, \text{m}^3 \quad \rightarrow \quad \delta V = 4.0\%
\]

\[
\rho = \frac{m}{V} = \frac{20 \, \text{kg}}{5 \, \text{m}^3} = 4 \, \text{kg/m}^3
\]

\[
\delta \rho = \delta m + \delta V = 6.5\% \quad \rightarrow \quad \Delta \rho = \delta \rho \cdot \rho = 0.065 \cdot 4 \, \text{kg/m}^3 = 0.26 \, \text{kg/m}^3
\]

\[
\rho = (4.0 \pm 0.26) \, \text{kg/m}^3
\]

\section{Erhaltungsgrösse Energie}



		\end{multicols*}{3}

\end{landscape}
\end{document}